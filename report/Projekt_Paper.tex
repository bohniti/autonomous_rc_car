\documentclass[journal]{IEEEtran}

\renewcommand\thesection{\arabic{section}} 
\renewcommand\thesubsectiondis{\thesection.\arabic{subsection}}
\renewcommand\thesubsubsectiondis{\thesubsectiondis.\alph{subsubsection}}
\renewcommand\theparagraphdis{\arabic{paragraph}.}

%\usepackage[retainorgcmds]{IEEEtrantools}
%\usepackage{bibentry}  
\usepackage{xcolor,soul,framed} %,caption

\colorlet{shadecolor}{yellow}
% \usepackage{color,soul}
\usepackage[pdftex]{graphicx}
\graphicspath{{../pdf/}{../jpeg/}}
\DeclareGraphicsExtensions{.pdf,.jpeg,.png}

\usepackage[cmex10]{amsmath}
%Mathabx do not work on ScribTex => Removed
%\usepackage{mathabx}
\usepackage{array}
\usepackage{mdwmath}
\usepackage{mdwtab}
\usepackage{eqparbox}
\usepackage{url}
\hyphenation{op-tical net-works semi-conduc-tor}
\usepackage{graphicx}



%\bstctlcite{IEEE:BSTcontrol}
%=== TITLE & AUTHORS ====================================================================
\begin{document}

\bstctlcite{IEEEexample:BSTcontrol}
	
    \title{    
    \includegraphics[width=4.5in]{photo/logo.png}
    \newline \newline
     IT-Projekt \\ 
     \textit{Autonmously driving Remote Control Car}
     }

  \author{
      \textbf{L\"OHR} Tim,
      \textbf{BOHNSTEDT} Timo, 
      \textbf{PALPANES} Ioannis
}

% The paper headers
\markboth{IT-Project at the TH-OHM
}{Roberg \MakeLowercase{\textit{et al.}}}


% ====================================================================
\maketitle
% === ABSTRACT 
\begin{abstract}
Im Rahmen unseres IT-Projekts beschäftigen wir uns mit der Entwicklung eines autonom fahrendes ferngesteuertes Auto. 
ToDo
\end{abstract}

% === KEYWORDS 
\begin{IEEEkeywords}
\hl{Machine Learning, IT-Project, RC-Car, RaspberryPi, Autonomusly driving}
\end{IEEEkeywords}
\IEEEpeerreviewmaketitle

% === I. Project background and motivation
\section{Project background and motivation}
ToDo

\section{RC Car}
\subsection{Technischer Aufbau}
\subsection{Servomotor}
\subsection{Verkabelung}

\section{Mikrocontroller}
\subsection{Raspberry Pi}
\subsubsection{Technischer Aufbau}
\subsubsection{Verkabelung}
\subsubsection{Kameramodul}

\subsection{Servodriver}
\subsubsection{Technischer Aufbau}
\subsubsection{Verkabelung}

\section{Webserver}
\subsection{Technischer Aufbau}
\subsection{Übertragung}

\section{Machinelles Lernen}
\subsection{Keras}
\subsection{Neuronales Netz}

\section{Model Training}
\subsection{Unity Demo}
\subsection{Bilder}

\section{Evaluation}

\section{Conclusion}

\section*{Acknowledgment}
\noindent The authors would like to thank Prof Dr. Florian Gallwitz from the University of Applied Science - Georg Simon OHM in Nuremberg for a really good supervising of our group. 



% if have a single appendix:
%\appendix[Proof of the Zonklar Equations]
% or
%\appendix  % for no appendix heading
% do not use \section anymore after \appendix, only \section*
% is possibly needed

% use appendices with more than one appendix
% then use \section to start each appendix
% you must declare a \section before using any
% \subsection or using \label (\appendices by itself
% starts a section numbered zero.)
%

% ============================================
%\appendices
%\section{Proof of the First Zonklar Equation}
%Appendix one text goes here %\cite{Roberg2010}.

% you can choose not to have a title for an appendix
% if you want by leaving the argument blank
%\section{}
%Appendix two text goes here.


% use section* for acknowledgement
%\section*{Acknowledgment}


%The authors would like to thank D. Root for the loan of the SWAP. The SWAP that can ONLY be usefull in Boulder...


% Can use something like this to put references on a page
% by themselves when using endfloat and the captionsoff option.
\ifCLASSOPTIONcaptionsoff
  \newpage
\fi



% trigger a \newpage just before the given reference
% number - used to balance the columns on the last page
% adjust value as needed - may need to be readjusted if
% the document is modified later
%\IEEEtriggeratref{8}
% The "triggered" command can be changed if desired:
%\IEEEtriggercmd{\enlargethispage{-5in}}

% ====== REFERENCE SECTION

%\begin{thebibliography}{1}

% IEEEabrv,

\bibliographystyle{IEEEtran}
\bibliography{Bibliography}
%\end{thebibliography}
% biography section
% 
% If you have an EPS/PDF photo (graphicx package needed) extra braces are
% needed around the contents of the optional argument to biography to prevent
% the LaTeX parser from getting confused when it sees the complicated
% \includegraphics command within an optional argument. (You could create
% your own custom macro containing the \includegraphics command to make things
% simpler here.)
%\begin{biography}[{\includegraphics[width=1in,height=1.25in,clip,keepaspectratio]{mshell}}]{Michael Shell}
% or if you just want to reserve a space for a photo:

% ==== SWITCH OFF the BIO for submission
% ==== SWITCH OFF the BIO for submission


%% if you will not have a photo at all:
%\begin{IEEEbiographynophoto}{Ignacio Ramos}
%(S'12) received the B.S. degree in electrical engineering from the University of Illinois at Chicago in 2009, and is currently working toward the Ph.D. degree at the University of Colorado at Boulder. From 2009 to 2011, he was with the Power and Electronic Systems Department at Raytheon IDS, Sudbury, MA. His research interests include high-efficiency microwave power amplifiers, microwave DC/DC converters, radar systems, and wireless power transmission.
%\end{IEEEbiographynophoto}

%% insert where needed to balance the two columns on the last page with
%% biographies
%%\newpage

%\begin{IEEEbiographynophoto}{Jane Doe}
%Biography text here.
%\end{IEEEbiographynophoto}
% ==== SWITCH OFF the BIO for submission
% ==== SWITCH OFF the BIO for submission



% You can push biographies down or up by placing
% a \vfill before or after them. The appropriate
% use of \vfill depends on what kind of text is
% on the last page and whether or not the columns
% are being equalized.

\vfill

% Can be used to pull up biographies so that the bottom of the last one
% is flush with the other column.
%\enlargethispage{-5in}



% that's all folks
\end{document}